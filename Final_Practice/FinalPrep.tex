\documentclass[letterpaper, 11pt]{article}
\usepackage{nopageno} % For removing page numbers
\usepackage[utf8]{inputenc}
\usepackage{color, colortbl} % For table coloring
\usepackage{titling} % For positioning of title preamble
\usepackage[margin=0.75in]{geometry} % For margin width setting
\usepackage{comment} % For block commenting
\usepackage{float} % For table positioning
% For math equation formatting
\usepackage{amsmath, amssymb, amsfonts}
\newcommand{\PMod}[1]{\ (\mathrm{mod}\ #1)}
\newcommand{\Mod}[1]{\ \mathrm{mod}\ #1}
\usepackage{parskip} % For automatic paragraph spacing/formatting
\usepackage{relsize} % For increased math mode font sizing
% For code blocks
\usepackage[dvipsnames,table]{xcolor}
\usepackage[most]{tcolorbox}
\usepackage{lmodern}
\renewcommand{\ttdefault}{lmtt}
\usepackage{listings, minted}
\lstset{
    basicstyle=\ttfamily\footnotesize,
    keepspaces=false,
    showstringspaces=false,
    keywordstyle=\color{blue},
    commentstyle=\sffamily\itshape\color{Green}\scriptsize,
    stringstyle = \color{red},
    breaklines=true,
    breakatwhitespace=false,
    tabsize=2
}
\tcbset{
    colback=gray!5!white,
    colframe=gray!75!black,
    parbox=false,
}
\setlength{\extrarowheight}{2pt}
\usepackage{titlesec} % Custom styling for section titles
\titleformat{\section}
  {\normalfont\LARGE\bfseries}{\thesection}{1em}{}
\titleformat{\subsection}
  {\normalfont}{\thesection}{1em}{}
% For side by side figures
\usepackage{multicol}
\usepackage{makecell}
% For horizontal lists
\usepackage{enumitem, tasks, varwidth}
% For custom page numbers
\usepackage{fancyhdr, lastpage}
\pagestyle{fancy}
\fancyhead{}
\fancyfoot{}
\renewcommand{\headrulewidth}{0pt}
\usepackage[skip=2pt]{caption}
\usepackage{graphicx}
\graphicspath{{../Images/}}

% Move title area to the top of the page
\setlength{\droptitle}{-4em}
\addtolength{\droptitle}{-4pt} 
\renewcommand{\arraystretch}{1.25}
% Disable paragraph indenting
\setlength{\parindent}{0pt}

\usepackage[none]{hyphenat}
\usepackage{times}
\usepackage{soul}

\title{Final Prep}
\author{}
\date{}

\begin{document}

\maketitle

\vspace{-4em}

\section*{Question 1}

Given the following DB schema:
\begin{itemize}
    \item \textit{Students} (\ul{\texttt{sid}: int}, \texttt{name}: string, \texttt{city}: string, \texttt{state}: string, \texttt{age}: int)
    \item \textit{Rents} (\ul{\texttt{cid}: int, \texttt{sid}: int}, \texttt{price}: int, \texttt{rentdate}: date)
    \item \textit{Cars} (\ul{\texttt{cid}: int}, \texttt{make}: string, \texttt{model}: string, \texttt{myear}: int)
\end{itemize}

\begin{enumerate}[leftmargin=*]
    \item Write the SQL statement that extracts how many students are in state MA.
    \item Write the SQL statement that extracts how many students are in each state.
    \item Write the SQL statement that finds the id and name of students who rented both a Honda Accord and a Toyota Prius. 
    \item Write the SQL statement that finds the id and name of students who only rented a Toyota Prius.
    \item Write the SQL Statement that extracts the students who rented all cars.
    \item Write the SQL statement to create a view called \textit{Cars2020} that contains information about all cars manufactured in 2020.
    \item Write the SQL to create a \textit{Cars} table with the constraints that all cars must be newer than 2010 (including 2010).
    \item Write the SQL to extract the average age of students by state, for those states that have at least 100 students.
    \item Write the SQL statement to extract the number of cars each student whose name starts with the letter `a’  rented. Make the query case insensitive with regard to the name of the student.
    \item Write the SQL statement to extract the id and name of students who only rented cars in 2018.
    \item Write the relational algebra to extract all Honda Accord cars older than 2018.
    \begin{tcolorbox}[oversize]
    \[\sigma_{(\textit{make}=\textit{`Honda'}) \wedge (\textit{model}=\textit{`Accord'}) \wedge (\textit{myear} < 2018)}\textit{Cars}\]
    \end{tcolorbox}
    \item Write the relational algebra to extract only the cid for all Honda Accord cars older than 2018.
    \begin{tcolorbox}[oversize]
    \[\pi_{\textit{cid}}(\sigma_{(\textit{make}=\textit{`Honda'}) \wedge (\textit{model}=\textit{`Accord'}) \wedge (\textit{myear} < 2018)}\textit{Cars})\]
    \end{tcolorbox}
    \item Write the relational algebra to extract the name of students who rented cards Hondas only in 2018.
    \begin{tcolorbox}[oversize]
    \[\rho(\textit{StudentsRent2018}, \pi_{\textit{sid,name}}(\textit{Students} \bowtie (\sigma_{\textit{make}=\textit{`Honda'}}\textit{Cars}) \bowtie (\sigma_{\textit{rentdate}=2018}\textit{Rents})))\]
    \[\rho(\textit{StudentsRentOther}, \pi_{\textit{sid,name}}(\textit{Students} \bowtie (\sigma_{\textit{make}=\textit{`Honda'}}\textit{Cars}) \bowtie (\sigma_{\textit{rentdate}!=2018}\textit{Rents})))\]
    \[\pi_{\textit{name}}(\textit{StudentsRent2018} - \textit{StudentsRentOther})\]
    \end{tcolorbox}
\end{enumerate}

\section*{Question 2}

Explain whether this relation is in BCNF. If not, decompose it to BCNF.
\begin{itemize}
    \item CSJDPQV
    \item Key is C
    \item Functional dependencies are: PQ $\to$ V, S $\to$ D
\end{itemize}

\begin{tcolorbox}
Check condition S $\to$ D
\begin{itemize}
    \item D is not included or equal to S
    \item S is not part of a key
\end{itemize}

S $\to$ D violates BCNF so we must decompose it.

Check condition PQ $\to$ V
\begin{itemize}
    \item V is not included or equal to PQ
    \item PQ is not included in a key
\end{itemize}

PQ $\to$ V violates BCNF so we must decompose it.

Decomposition of CSJDPQV

Decompose for PQ $\to$ V (PQV, CSJDPQ)

Decompose for  S $\to$ D (PQV, SD, CSJPQ)

Check if we lost any dependency. If yes, it means we need to try to do 3NF. In this case, no dependency was lost.
\end{tcolorbox}

\end{document}
