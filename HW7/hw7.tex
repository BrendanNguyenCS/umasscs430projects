\documentclass[letterpaper, 11pt]{article}
\usepackage{nopageno} % For removing page numbers
\usepackage[utf8]{inputenc}
\usepackage{color, colortbl} % For table coloring
\usepackage{titling} % For positioning of title preamble
\usepackage[margin=0.75in]{geometry} % For margin width setting
\usepackage{comment} % For block commenting
\usepackage{float} % For table positioning
% For math equation formatting
\usepackage{amsmath, amssymb, amsfonts}
\newcommand{\PMod}[1]{\ (\mathrm{mod}\ #1)}
\newcommand{\Mod}[1]{\ \mathrm{mod}\ #1}
\usepackage{parskip} % For automatic paragraph spacing/formatting
\usepackage{relsize} % For increased math mode font sizing
% For code blocks
\usepackage[dvipsnames,table]{xcolor}
\usepackage[most]{tcolorbox}
\usepackage{lmodern}
\renewcommand{\ttdefault}{lmtt}
\usepackage{listings, minted}
\lstset{
    basicstyle=\ttfamily\footnotesize,
    keepspaces=false,
    showstringspaces=false,
    keywordstyle=\color{blue},
    commentstyle=\sffamily\itshape\color{Green}\scriptsize,
    stringstyle = \color{red},
    breaklines=true,
    breakatwhitespace=false,
    tabsize=2
}
\tcbset{
    colback=gray!5!white,
    colframe=gray!75!black,
    oversize,
}
\setlength{\extrarowheight}{2pt}
\usepackage{titlesec} % Custom styling for section titles
\titleformat{\section}
  {\normalfont\LARGE\bfseries}{\thesection}{1em}{}
\titleformat{\subsection}
  {\normalfont}{\thesection}{1em}{}
% For side-by-side figures
\usepackage{multicol}
\usepackage{makecell}
% For horizontal lists
\usepackage{enumitem, tasks, varwidth}
% For custom page numbers
\usepackage{fancyhdr, lastpage}
\pagestyle{fancy}
\fancyhead{}
\fancyfoot{}
\renewcommand{\headrulewidth}{0pt}
\usepackage[skip=2pt]{caption}
\usepackage{graphicx}
\graphicspath{{../Images/}}

% Move title area to the top of the page
\setlength{\droptitle}{-4em}
\addtolength{\droptitle}{-4pt} 
\renewcommand{\arraystretch}{1.25}
% Disable paragraph indenting
\setlength{\parindent}{0pt}

\usepackage[none]{hyphenat}
\usepackage{times}
\usepackage{soul}

\title{CS430 Homework 7}
\author{Brendan Nguyen}
\date{Due: Sunday, May 7, 2023}

\begin{document}

\maketitle

\section*{Question 1 (80 points)}

Given the relation with the following attributes ACFSBDM.

Note that C is the key. This relation has this set of functional dependencies: FS $\to$ B, A $\to$ D.

\begin{enumerate}[label={\alph*}),leftmargin=*]
    \item Explain why this relation is not in BCNF.
    \item Decompose this relation into BCNF.
\end{enumerate}

\section*{Question 2 (20 points)}

Write the SQL statements to:
\begin{enumerate}[label={\alph*}),leftmargin=*]
    \item Create user joe with password joe123.

\begin{tcolorbox}
\begin{lstlisting}[language=SQL]
CREATE USER joe IDENTIFIED BY joe123;
\end{lstlisting}
\end{tcolorbox}

    \item Write the SQL statement to give \texttt{SELECT} and \texttt{INSERT} access to user joe to table Sailors. Give access such that user joe can further give access to other users. 

\begin{tcolorbox}
\begin{lstlisting}[language=SQL]
GRANT SELECT, INSERT ON Sailors TO joe WITH ADMIN OPTION;
\end{lstlisting}
\end{tcolorbox}
\end{enumerate}


\end{document}
