\documentclass[letterpaper, 11pt]{article}
\usepackage{nopageno} % For removing page numbers
\usepackage[utf8]{inputenc}
\usepackage{color, colortbl} % For table coloring
\usepackage{titling} % For positioning of title preamble
\usepackage[margin=0.75in]{geometry} % For margin width setting
\usepackage{comment} % For block commenting
\usepackage{float} % For table positioning
% For math equation formatting
\usepackage{amsmath, amssymb, amsfonts}
\newcommand{\PMod}[1]{\ (\mathrm{mod}\ #1)}
\newcommand{\Mod}[1]{\ \mathrm{mod}\ #1}
\usepackage{parskip} % For automatic paragraph spacing/formatting
\usepackage{relsize} % For increased math mode font sizing
% For code blocks
\usepackage[dvipsnames,table]{xcolor}
\usepackage[most]{tcolorbox}
\usepackage{lmodern}
\renewcommand{\ttdefault}{lmtt}
\usepackage{listings, minted}
\lstset{
    basicstyle=\ttfamily\footnotesize,
    keepspaces=false,
    showstringspaces=false,
    keywordstyle=\color{blue},
    commentstyle=\sffamily\itshape\color{Green}\scriptsize,
    stringstyle = \color{red},
    breaklines=true,
    breakatwhitespace=false,
    tabsize=2
}
\tcbset{
    colback=gray!5!white,
    colframe=gray!75!black,
    oversize,
}
\setlength{\extrarowheight}{2pt}
\usepackage{titlesec} % Custom styling for section titles
\titleformat{\section}
  {\normalfont\LARGE\bfseries}{\thesection}{1em}{}
\titleformat{\subsection}
  {\normalfont}{\thesection}{1em}{}
% For side by side figures
\usepackage{multicol}
\usepackage{makecell}
% For horizontal lists
\usepackage{enumitem, tasks, varwidth}
% For custom page numbers
\usepackage{fancyhdr, lastpage}
\pagestyle{fancy}
\fancyhead{}
\fancyfoot{}
\renewcommand{\headrulewidth}{0pt}
\usepackage[skip=2pt]{caption}
\usepackage{graphicx}
\graphicspath{{../Images/}}

% Move title area to the top of the page
\setlength{\droptitle}{-4em}
\addtolength{\droptitle}{-4pt} 
\renewcommand{\arraystretch}{1.25}
% Disable paragraph indenting
\setlength{\parindent}{0pt}

\usepackage[none]{hyphenat}
\usepackage{times}
\usepackage{soul}

\title{CS430 Homework 1}
\author{Brendan Nguyen}
\date{Due: Wednesday, Feb. 15, 2023}

\begin{document}

\maketitle

\section*{Question 1 (30 points)}

Consider a database schema with three relations:
\begin{itemize}
    \item \textit{Books} (\ul{\texttt{bid}: int}, \texttt{bname}: string, \texttt{author}: string, \texttt{pubyear}: int, \texttt{pubcompany}: string)
    \item \textit{Students} (\ul{\texttt{sid}: int}, \texttt{sname}: string, \texttt{age}: real, \texttt{state}: string)
    \item \textit{Reads} (\ul{\texttt{sid}: int, \texttt{bid}: int}, \texttt{year}: int)
\end{itemize}

The primary keys are underlined in each relation. A book is uniquely identified by \texttt{bid}. A student is uniquely identified by \texttt{sid}. If a student reads a book, a record will be present in the \textit{Reads} relation, with that \texttt{sid} and \texttt{bid} and the year the book was read. The following relations instances are also given:

\begin{table}[H]
    \centering
    \caption*{Students}
    \begin{tabular}{|l|l|l|l|}
    \hline
        \cellcolor[HTML]{b4c6e7} sid & \cellcolor[HTML]{b4c6e7} sname & \cellcolor[HTML]{b4c6e7} age & \cellcolor[HTML]{b4c6e7} state \\
        \hline
        20 & mary & 21 & MA \\
        \hline
        10 & anne & 20 & NY \\
        \hline
        30 & joe & 21 & MA \\
        \hline
        40 & mary & 21 & VT \\
        \hline
        60 & linda & 23 & MA \\
        \hline
    \end{tabular}
\end{table}
\vspace{-2em}
\begin{minipage}[t]{0.7\textwidth}
\begin{table}[H]
    \centering
    \caption*{Books}
    \begin{tabular}{|l|l|l|l|l|}
    \hline
        \cellcolor[HTML]{b4c6e7} bid & \cellcolor[HTML]{b4c6e7} bname & \cellcolor[HTML]{b4c6e7} author & \cellcolor[HTML]{b4c6e7} pubyear & \cellcolor[HTML]{b4c6e7} pubcompany \\
        \hline
        102 & ulysses & joyce & 1920 & simon \\
        \hline
        101 & lord of the rings & tolkien & 1954 & alien \\
        \hline
        103 & other book & joyce & 1920 & penguin \\
        \hline
    \end{tabular}
\end{table}
\end{minipage}
\begin{minipage}[t]{0.2\textwidth}
\begin{table}[H]
    \centering
    \caption*{Reads}
    \begin{tabular}{|l|l|l|}
        \hline
        \cellcolor[HTML]{b4c6e7} sid & \cellcolor[HTML]{b4c6e7} bid & \cellcolor[HTML]{b4c6e7} year \\
        \hline
        20 & 101 & 2020 \\
        \hline
        20 & 102 & 2021 \\
        \hline
        30 & 103 & 2020 \\
        \hline
    \end{tabular}
\end{table}
\end{minipage}
\vspace{1em}

Using the relation instances from above, show the resulted relation for each of the following relational algebra expressions:
\begin{enumerate}[label={\alph*})]
    \item $\sigma_{\textit{author='joyce'}}\textit{Books}$
    \begin{table}[H]
        \centering
        \begin{tabular}{|l|l|l|l|l|}
        \hline
            \cellcolor[HTML]{b4c6e7} bid & \cellcolor[HTML]{b4c6e7} bname & \cellcolor[HTML]{b4c6e7} author & \cellcolor[HTML]{b4c6e7} pubyear & \cellcolor[HTML]{b4c6e7} pubcompany \\
            \hline
            102 & ulysses & joyce & 1920 & simon \\
            \hline
            103 & other book & joyce & 1920 & penguin \\
            \hline
        \end{tabular}
    \end{table}
    
    \item $\pi_{\textit{author,pubyear}}(\sigma_{\textit{author='joyce'}}\textit{Books})$
    \begin{table}[H]
        \centering
        \begin{tabular}{|l|l|}
        \hline
            \cellcolor[HTML]{b4c6e7} author & \cellcolor[HTML]{b4c6e7} pubyear \\
            \hline
            ulysses & joyce \\
            \hline
            other book & joyce \\
            \hline
        \end{tabular}
    \end{table}
    
    \item $(\sigma_{\textit{sname='mary'}}\textit{Students}) \bowtie \textit{Reads}$
    \begin{table}[H]
        \centering
        \begin{tabular}{|l|l|l|l|l|l|}
            \hline
            \cellcolor[HTML]{b4c6e7} sid & \cellcolor[HTML]{b4c6e7} sname & \cellcolor[HTML]{b4c6e7} age & \cellcolor[HTML]{b4c6e7} state & \cellcolor[HTML]{b4c6e7} bid & \cellcolor[HTML]{b4c6e7} year \\
            \hline
            20 & mary & 21 & MA & 101 & 2020 \\
            \hline
            20 & mary & 21 & VT & 102 & 2021 \\
            \hline
        \end{tabular}
    \end{table}
    \item $(\sigma_{\textit{state='MA'}}\textit{Students}) \times (\sigma_{\textit{year=2020}}\textit{Reads})$
    \begin{table}[H]
        \centering
        \begin{tabular}{|l|l|l|l|l|l|l|}
        \hline
            \cellcolor[HTML]{b4c6e7} (sid) & \cellcolor[HTML]{b4c6e7} sname & \cellcolor[HTML]{b4c6e7} age & \cellcolor[HTML]{b4c6e7} state & \cellcolor[HTML]{b4c6e7} (sid) & \cellcolor[HTML]{b4c6e7} bid & \cellcolor[HTML]{b4c6e7} year \\
            \hline
            20 & mary & 21 & MA & 20 & 101 & 2020 \\
            \hline
            20 & mary & 21 & MA & 30 & 103 & 2020 \\
            \hline
            30 & joe & 21 & MA & 20 & 101 & 2020 \\
            \hline
            30 & joe & 21 & MA & 30 & 103 & 2020 \\
            \hline
            60 & linda & 23 & MA & 20 & 101 & 2020 \\
            \hline
            60 & linda & 23 & MA & 30 & 103 & 2020 \\
            \hline
        \end{tabular}
    \end{table}
    \item $\rho(A(\textit{bname} \to \textit{name}), \sigma_{(\textit{pubcompany='simon})\vee (\textit{pubcompany='alien'})} \textit{Books})$
    \begin{table}[H]
        \centering
        \begin{tabular}{|l|l|l|l|l|}
        \hline
            \cellcolor[HTML]{b4c6e7} bid & \cellcolor[HTML]{b4c6e7} name & \cellcolor[HTML]{b4c6e7} author & \cellcolor[HTML]{b4c6e7} pubyear & \cellcolor[HTML]{b4c6e7} pubcompany \\
            \hline
            102 & ulysses & joyce & 1920 & simon \\
            \hline
            101 & lord of the rings & tolkien & 1954 & alien \\
            \hline
        \end{tabular}
    \end{table}
    \item $\left[\text{CS630 only}\right]$ $\textit{Students} \bowtie \textit{Reads} \bowtie (\sigma_{\textit{pubyear \textless 1950}}\textit{Books})$
    \begin{table}[H]
        \centering
        \begin{tabular}{|l|l|l|l|l|l|l|l|l|l|}
            \hline
            \cellcolor[HTML]{b4c6e7} sid & \cellcolor[HTML]{b4c6e7} sname & \cellcolor[HTML]{b4c6e7} age & \cellcolor[HTML]{b4c6e7} state & \cellcolor[HTML]{b4c6e7} bid & \cellcolor[HTML]{b4c6e7} year & \cellcolor[HTML]{b4c6e7} bname & \cellcolor[HTML]{b4c6e7} author & \cellcolor[HTML]{b4c6e7} pubyear & \cellcolor[HTML]{b4c6e7} pubcompany \\
            \hline
            30 & joe & 21 & MA & 103 & 2020 & other book & joyce & 1920 & penguin \\
            \hline
            40 & mary & 21 & VT & 102 & 2021 & ulysses & joyce & 1920 & simon \\
            \hline
        \end{tabular}
    \end{table}
\end{enumerate}

\section*{Question 2 (40 points)}

Consider a database schema with three relations:
\begin{itemize}
    \item \textit{Actors} (\ul{\texttt{aid}: int}, \texttt{aname}: string, \texttt{age}: real, \texttt{city}: string, \texttt{state}: string)
    \item \textit{Playsin} (\ul{\texttt{aid}: int, \texttt{mid}: int}, \texttt{character}: string)
    \item \textit{Movies} (\ul{\texttt{mid}: int}, \texttt{mname}: string, \texttt{year}: int, \texttt{studio}: string)
\end{itemize}

Primary keys are underlined in each relation. Attribute aid uniquely identifies an actor in \textit{Actors} relation. An actor has an id (\texttt{aid}), a name (attr. \texttt{aname}), an age (attr. \texttt{age}), and a city and state(attributes \texttt{city} and \texttt{state}). Attribute \texttt{mid} uniquely identifies a movie in relation \textit{Movies}. A movie has an id(\texttt{mid}), a name(attr. \texttt{mname}), a year (attr. \texttt{year}) and a studio that produced it (attr. \texttt{studio}). Relation \textit{Playsin} contains information about actors who played in movies. Attribute \texttt{character} is the name of the character played by the actor with \texttt{aid} when playing in \texttt{mid}.

Write relational algebra queries for the following queries:
\begin{enumerate}[label={\alph*})]
    \item Find the information about movies produced by `WB' or `Universal' studios
    \begin{tcolorbox}
    \[\sigma_{(\textit{studio=`WB'}) \vee (\textit{studio=`Universal'})}\textit{Movies}\]    
    \end{tcolorbox}
    \item Find the names of actors who are older than 25 and are from state VT.
    \begin{tcolorbox}
    \[\pi_{\textit{aname}}(\sigma_{(\textit{age \textgreater 25}) \wedge (\textit{state = `VT'})}\textit{Actors})\]
    \end{tcolorbox}    
    \item Find the names and ages of the actors who played only in movies only in 2015.
    \begin{tcolorbox}
    \[\pi_{\textit{aname,age}}(\textit{Actors} \bowtie \textit{Playsin} \bowtie (\sigma_{\textit{year=2015}}\textit{Movies}))\]
    \end{tcolorbox}    
    \item Find the names, age and city of actors who are from Boston MA and played some movies produced by `Universal' studio.
    \begin{tcolorbox}
    \[\pi_{\textit{aname,age,city}}((\sigma_{(\textit{city=`Boston'})\wedge (\textit{state=`MA'})}\textit{Actors}) \bowtie \textit{Playsin} \bowtie (\sigma_{\textit{studio=`Universal'}}\textit{Movies}))\]
    \end{tcolorbox}    
    \item Find the name and age of the actors who played in movies both in 2012 and 2018.
    \begin{tcolorbox}
    \[\pi_{\textit{aname,age}}((\sigma_{\textit{year=2012}}\textit{Movies} \bowtie \textit{Playsin}) \cap (\sigma_{\textit{year=2018}}\textit{Movies} \bowtie \textit{Playsin})) \bowtie \textit{Actors}\]
    \end{tcolorbox}    
    \item Find the names of the actors older than 30 who played in a movie produced by `WB' studio in 2018.
    \begin{tcolorbox}
    \[\pi_{\textit{aname}}((\sigma_{\textit{age \textgreater 30}}\textit{Actors}) \bowtie (\textit{Playsin} \bowtie (\sigma_{(studio=`WB') \wedge (\textit{year=2018})}\textit{Movies})))\]
    \end{tcolorbox}    
    \item Find the information about actors and movies they played in. The result should contain the name and age of actors and the name of movies.
    \begin{tcolorbox}
    \[\pi_{\textit{aname,age,mname}}(\textit{Actors} \bowtie \textit{Playsin} \bowtie \textit{Movies})\]
    \end{tcolorbox}    
    \item Find the names and ages of actors from MA who played as character `Batman'.
    \begin{tcolorbox}
    \[\pi_{\textit{aname,age}}(\sigma_{(\textit{state=`MA'})\wedge (\textit{character=`Batman'})}(\textit{Actors} \bowtie \textit{Playsin}))\]
    \end{tcolorbox}    
    \item $\left[\text{CS630 only}\right]$ Find the name and age of actors who played in movies produced by `Paramount' (in any year) and never played in any movie produced by `WB' in year 2020.
    \item $\left[\text{CS630 only}\right]$ Find names of movies in which actors from both MA and NY states played.
\end{enumerate}

\section*{Question 3 (30 points)}

Consider a database schema with three relations:
\begin{itemize}
    \item \textit{Books} (\ul{\texttt{bid}: int}, \texttt{bname}: string, \texttt{author}: string, \texttt{pubyear}: int, \texttt{pubcompany}: string)
    \item \textit{Students} (\ul{\texttt{sid}: int}, \texttt{sname}: string, \texttt{age}: real, \texttt{state}: string)
    \item \textit{Reads} (\ul{\texttt{sid}: int, \texttt{bid}: int}, \texttt{year}: int)
\end{itemize}

Primary keys are underlined  in each relation. A book is uniquely identified by \texttt{bid}. A book has an id (\texttt{bid}), a name (\texttt{bname}), one author (attribute \texttt{author}), a publication year (\texttt{pubyear}), and a publishing company (\texttt{pubcompany}). A student is uniquely identified by \texttt{sid}. A student has an id (\texttt{sid}), a name (attr. \texttt{sname}), age (attr. \texttt{age}) and a state (attr. \texttt{state}). If a student reads a book, a record will be present in the Reads relation, with that \texttt{sid} and \texttt{bid} and the \texttt{year} the book was read.

Write the relational algebra expressions for the following queries:
\begin{enumerate}[label={\alph*})]
    \item Find the information about the youngest students.
    \begin{tcolorbox}
    \[\rho(S1, \textit{Students})\]
    \[\rho(S2, \textit{Students})\]
    \[\rho(\textit{Temp}(1 \to \textit{fsid}, 2 \to \textit{fsname}, 3 \to \textit{fage}, 4 \to \textit{fstate}), S1 \bowtie_{\textit{S1.age \textgreater S2.age}}S2)\]
    \[\rho(\textit{TempLeft}, \pi_{\textit{fsid,fsname,fage,fstate}}\textit{Temp})\]
    \[\textit{Students} - \textit{TempLeft}\]
    \end{tcolorbox}
    \item Find the information about the books that are either published in 2010 or 2020.
    \begin{tcolorbox}
    \[\sigma_{(\textit{pubyear=2010}) \wedge (\textit{pubyear=2020})}\textit{Books}\]    
    \end{tcolorbox}
    \item Find the names, pub year and pub company of the oldest books.
    \begin{tcolorbox}
    \[\rho(B1, \textit{Books})\]
    \[\rho(B2, \textit{Books})\]
    \[\rho(\textit{Temp}(1 \to \textit{fbid}, 2 \to \textit{fbname}, 3 \to \textit{fauthor}, 4 \to \textit{fpubyear}, 5 \to \textit{fpubcompany}), B1 \bowtie_{\textit{B1.pubyear \textgreater B2.pubyear}} B2)\]
    \[\rho(\textit{TempLeft}, \pi_{\textit{fbname,fpubye\textit{}ar,fpubcompany}}\textit{Temp})\]
    \[\textit{Books} - \textit{TempLeft}\]
    \end{tcolorbox}
    \item Find the names of the students from MA who read some books both in 2015 and 2018.
    \begin{tcolorbox}
    \[\pi_{\textit{sname}}(((\sigma_{\textit{year=2015}}\textit{Reads}) \bowtie (\sigma_{\textit{state=`MA'}}\textit{Students})) \cap ((\sigma_{\textit{year=2018}}\textit{Reads}) \bowtie (\sigma_{\textit{state=`MA'}}\textit{Students})))\]
    \end{tcolorbox}
    \item Find the names of the books that were read by all students.
    \begin{tcolorbox}
    \[\pi_{\textit{bname}}((\pi_{\textit{bid,sid}}\textit{Reads} / \pi_{\textit{sid}}\textit{Students}) \bowtie \textit{Books})\]
    \end{tcolorbox}    
    \item $\left[\text{CS630 only}\right]$ Find the names, pubyear and pubcompany of the books that were read by all students from MA.
\end{enumerate}

\end{document}
