\documentclass[letterpaper, 11pt]{article}
\usepackage{nopageno} % For removing page numbers
\usepackage[utf8]{inputenc}
\usepackage{color, colortbl} % For table coloring
\usepackage{titling} % For positioning of title preamble
\usepackage[margin=0.75in]{geometry} % For margin width setting
\usepackage{comment} % For block commenting
\usepackage{float} % For table positioning
% For math equation formatting
\usepackage{amsmath, amssymb, amsfonts}
\newcommand{\PMod}[1]{\ (\mathrm{mod}\ #1)}
\newcommand{\Mod}[1]{\ \mathrm{mod}\ #1}
\usepackage{parskip} % For automatic paragraph spacing/formatting
\usepackage{relsize} % For increased math mode font sizing
% For code blocks
\usepackage[dvipsnames,table]{xcolor}
\usepackage[most]{tcolorbox}
\usepackage{lmodern}
\renewcommand{\ttdefault}{lmtt}
\usepackage{listings, minted}
\lstset{
    basicstyle=\ttfamily\footnotesize,
    keepspaces=false,
    showstringspaces=false,
    keywordstyle=\color{blue},
    commentstyle=\sffamily\itshape\color{Green}\scriptsize,
    stringstyle = \color{red},
    breaklines=true,
    breakatwhitespace=false,
    tabsize=2
}
\tcbset{
    colback=gray!5!white,
    left=0pt,
    right=0pt,
    top=0pt,
    bottom=0pt,
    colframe=gray!75!black,
}
\setlength{\extrarowheight}{2pt}
\usepackage{titlesec} % Custom styling for section titles
\titleformat{\section}
  {\normalfont\LARGE\bfseries}{\thesection}{1em}{}
\titleformat{\subsection}
  {\normalfont}{\thesection}{1em}{}
% For side-by-side figures
\usepackage{multicol}
\usepackage{makecell}
% For horizontal lists
\usepackage{enumitem, tasks, varwidth}
% For custom page numbers
\usepackage{fancyhdr, lastpage}
\pagestyle{fancy}
\fancyhead{}
\fancyfoot{}
\renewcommand{\headrulewidth}{0pt}
\usepackage[skip=2pt]{caption}
\usepackage{graphicx}
\graphicspath{{../Images/}}

% Move title area to the top of the page
\setlength{\droptitle}{-4em}
\addtolength{\droptitle}{-4pt} 
\renewcommand{\arraystretch}{1.25}
% Disable paragraph indenting
\setlength{\parindent}{0pt}

\usepackage[none]{hyphenat}
\usepackage{times}
\usepackage{soul}

\title{CS430 Homework 3}
\author{Brendan Nguyen (\#01763771)}
\date{Due: Wednesday, Mar. 15, 2023}

\begin{document}

\maketitle

\section*{Question 1 (40 points)}

Given the following db schema:
\begin{itemize}
    \item \textit{Customers} (\ul{\texttt{cid}: int}, \texttt{name}: string, \texttt{city}: string, \texttt{state}: string, \texttt{age}: int)
    \item \textit{Has\textunderscore account} (\ul{\texttt{cid}: int, \texttt{aid}: int}, \texttt{since}: date)
    \item \textit{Accounts} (\ul{\texttt{aid}: int}, \texttt{atype}: string, \texttt{amount}: real)
\end{itemize}

Primary keys are uld in each relation. Relation \textit{Customers} contains information about customers. A customer is uniquely identified by \texttt{cid}. Relation \textit{Accounts} contains information about bank accounts. An account is uniquely identified by \texttt{aid}. The type of the account is given in column \texttt{atype} (for example  \texttt{atype = `savings'}).  Customers have accounts. For a customer that has an account (i.e. is the owner of that account) there is a record in relation \textit{Has\_account}, that has the \texttt{cid} of that customer and the \texttt{aid} of that account, as well as the date when the account was opened (column \texttt{since}).

Using this db schema:
\begin{enumerate}[label={\alph*}),leftmargin=*]
    \item Write the SQL statements to create the tables from this db schema. Do not forget to define the necessary key constraints. The statements should be written in an order such that if executed in that order will not cause an error.
    \item Write SQL statements to insert two records in each table. The statements should be written in such an order such that if executed in that order do not cause an error.
    \item Write the SQL query that extracts the id, name of customers from city \texttt{`Boston'} that have some account with an amount smaller than 5000.  The result should not contain duplicates. Sort the result by name in descending order.
    \item Write the SQL query to extract the id, name, and age of customers who did not opened any account between Jan 1\textsuperscript{st} 2020 and Dec 1\textsuperscript{st} 2021 (including these dates).
    \item Write the SQL query to extract the id, name, and age of customers who have both \texttt{`savings'} and \texttt{`checking'} types of accounts.
    \item Write the SQL query to extract the name and id of customers who have at most 10,000 across all their accounts.
    \item Write the SQL query to extract all account ids of type checking that have at least 2 owners.
    \item Write the SQL to get the number of accounts each customer older than 25 and younger than 35 has. In the result include only customers who have at least 2 accounts.
    \item Write the SQL to extract the id, name, and age of customers who opened accounts both in year 2018 and 2020.
    \item Write the SQL query to extract the id and the name of the customers who are from \texttt{`MA'} state and who have at least 2 accounts of type \texttt{`savings'}. 
\end{enumerate}

\section*{Question 2 (60 points)}

Given the following db schema:
\begin{itemize}
    \item \textit{Articles} (\ul{\texttt{aid}: int}, \texttt{aname}: string, \texttt{first\textunderscore author}: string, \texttt{pubyear}: int, \texttt{pubcompany}: string)
    \item \textit{Students} (\ul{\texttt{sid}: int}, \texttt{sname}: string, \texttt{age}: real, \texttt{state}: string)
    \item \textit{Reads} (\ul{\texttt{sid}: int, \texttt{aid}: int}, \texttt{year}: int) 
\end{itemize}

Primary keys are uld in each relation. An article is uniquely identified by \texttt{aid}. An article has an id (\texttt{aid}), a name (\texttt{aname}), one first author (\texttt{first\textunderscore author}), a publication year (\texttt{pubyear}), and a publishing company (\texttt{pubcompany}). A student is uniquely identified by \texttt{sid}. A student has an id (\texttt{sid}), a name (attr. \texttt{sname}), age (attr. \texttt{age}) and a state (attr. \texttt{state}). If a student reads an article, a record will be present in the \textit{Reads} relation, with that \texttt{sid}, \texttt{aid}, and the \texttt{year} the article was read.

For this db schema:
\begin{enumerate}[label={\alph*}),leftmargin=*]
    \item Write the SQL statements to create the tables from this db schema. Do not forget to define the key constraints. The statements should be written in an order such that if executed in that order will not cause an error.
    \item For each table, write an insert statement to insert one record. The statements should be written in an order such that if executed in that order will not cause an error.
    \item Write the SQL statement to find the number of articles that have a \texttt{first\textunderscore author} whose name contain string \texttt{`an'}. The query has to be case insensitive with regards string \texttt{`an'}.
    \item Write the SQL statement to find information about the newest articles (hint: \texttt{pubyear} is max). Sort the result by the name of the article in ascending order.
    \item Write the SQL statement to find the id, name and age of students who read all articles. The result should contain no duplicates.
    \item Write the SQL statement to find the id and name of students who read some articles published in 2020 and did not read any article published in 2018. The result should be sorted by the name of the students in descending order.
    \item Write the SQL statement to find the ids and names of the students who read some articles in the same year when that article was published.
    \item Write the SQL that the number of articles read by each student. The result should contain information only about those students who read at least 3 articles.
    \item Write the SQL statement that extracts the minimum, maximum and average age of students for each state. Show this information only for states where there are at least 2 students.
    \item Write the SQL statement to find the articles whose first author name (\texttt{first\textunderscore author}) starts with letter \texttt{`B'} and that are published either before 2018 or after 2020 (note that 2018 and 2020 are included).
    \item Write the SQL statement to find the id, name of the students who read all articles published in year 2022 by the publishing company \texttt{`penguin'}. The result should contain no duplicates.
    \item Find the id, name, age, and state of the students who did not read all articles published by \texttt{`simon'} publishing company. The result should contain no duplicates. 
\end{enumerate}

\end{document}
